\documentclass[12pt,a4paper]{article}

\usepackage[a4paper,text={16.5cm,25.2cm},centering]{geometry}
\usepackage{lmodern}
\usepackage{amssymb,amsmath}
\usepackage{bm}
\usepackage{graphicx}
\usepackage{microtype}
\usepackage{hyperref}
\setlength{\parindent}{0pt}
\setlength{\parskip}{1.2ex}

\hypersetup
       {   pdfauthor = { Marco Fasondini },
           pdftitle={ foo },
           colorlinks=TRUE,
           linkcolor=black,
           citecolor=blue,
           urlcolor=blue
       }




\usepackage{upquote}
\usepackage{listings}
\usepackage{xcolor}
\lstset{
    basicstyle=\ttfamily\footnotesize,
    upquote=true,
    breaklines=true,
    breakindent=0pt,
    keepspaces=true,
    showspaces=false,
    columns=fullflexible,
    showtabs=false,
    showstringspaces=false,
    escapeinside={(*@}{@*)},
    extendedchars=true,
}
\newcommand{\HLJLt}[1]{#1}
\newcommand{\HLJLw}[1]{#1}
\newcommand{\HLJLe}[1]{#1}
\newcommand{\HLJLeB}[1]{#1}
\newcommand{\HLJLo}[1]{#1}
\newcommand{\HLJLk}[1]{\textcolor[RGB]{148,91,176}{\textbf{#1}}}
\newcommand{\HLJLkc}[1]{\textcolor[RGB]{59,151,46}{\textit{#1}}}
\newcommand{\HLJLkd}[1]{\textcolor[RGB]{214,102,97}{\textit{#1}}}
\newcommand{\HLJLkn}[1]{\textcolor[RGB]{148,91,176}{\textbf{#1}}}
\newcommand{\HLJLkp}[1]{\textcolor[RGB]{148,91,176}{\textbf{#1}}}
\newcommand{\HLJLkr}[1]{\textcolor[RGB]{148,91,176}{\textbf{#1}}}
\newcommand{\HLJLkt}[1]{\textcolor[RGB]{148,91,176}{\textbf{#1}}}
\newcommand{\HLJLn}[1]{#1}
\newcommand{\HLJLna}[1]{#1}
\newcommand{\HLJLnb}[1]{#1}
\newcommand{\HLJLnbp}[1]{#1}
\newcommand{\HLJLnc}[1]{#1}
\newcommand{\HLJLncB}[1]{#1}
\newcommand{\HLJLnd}[1]{\textcolor[RGB]{214,102,97}{#1}}
\newcommand{\HLJLne}[1]{#1}
\newcommand{\HLJLneB}[1]{#1}
\newcommand{\HLJLnf}[1]{\textcolor[RGB]{66,102,213}{#1}}
\newcommand{\HLJLnfm}[1]{\textcolor[RGB]{66,102,213}{#1}}
\newcommand{\HLJLnp}[1]{#1}
\newcommand{\HLJLnl}[1]{#1}
\newcommand{\HLJLnn}[1]{#1}
\newcommand{\HLJLno}[1]{#1}
\newcommand{\HLJLnt}[1]{#1}
\newcommand{\HLJLnv}[1]{#1}
\newcommand{\HLJLnvc}[1]{#1}
\newcommand{\HLJLnvg}[1]{#1}
\newcommand{\HLJLnvi}[1]{#1}
\newcommand{\HLJLnvm}[1]{#1}
\newcommand{\HLJLl}[1]{#1}
\newcommand{\HLJLld}[1]{\textcolor[RGB]{148,91,176}{\textit{#1}}}
\newcommand{\HLJLs}[1]{\textcolor[RGB]{201,61,57}{#1}}
\newcommand{\HLJLsa}[1]{\textcolor[RGB]{201,61,57}{#1}}
\newcommand{\HLJLsb}[1]{\textcolor[RGB]{201,61,57}{#1}}
\newcommand{\HLJLsc}[1]{\textcolor[RGB]{201,61,57}{#1}}
\newcommand{\HLJLsd}[1]{\textcolor[RGB]{201,61,57}{#1}}
\newcommand{\HLJLsdB}[1]{\textcolor[RGB]{201,61,57}{#1}}
\newcommand{\HLJLsdC}[1]{\textcolor[RGB]{201,61,57}{#1}}
\newcommand{\HLJLse}[1]{\textcolor[RGB]{59,151,46}{#1}}
\newcommand{\HLJLsh}[1]{\textcolor[RGB]{201,61,57}{#1}}
\newcommand{\HLJLsi}[1]{#1}
\newcommand{\HLJLso}[1]{\textcolor[RGB]{201,61,57}{#1}}
\newcommand{\HLJLsr}[1]{\textcolor[RGB]{201,61,57}{#1}}
\newcommand{\HLJLss}[1]{\textcolor[RGB]{201,61,57}{#1}}
\newcommand{\HLJLssB}[1]{\textcolor[RGB]{201,61,57}{#1}}
\newcommand{\HLJLnB}[1]{\textcolor[RGB]{59,151,46}{#1}}
\newcommand{\HLJLnbB}[1]{\textcolor[RGB]{59,151,46}{#1}}
\newcommand{\HLJLnfB}[1]{\textcolor[RGB]{59,151,46}{#1}}
\newcommand{\HLJLnh}[1]{\textcolor[RGB]{59,151,46}{#1}}
\newcommand{\HLJLni}[1]{\textcolor[RGB]{59,151,46}{#1}}
\newcommand{\HLJLnil}[1]{\textcolor[RGB]{59,151,46}{#1}}
\newcommand{\HLJLnoB}[1]{\textcolor[RGB]{59,151,46}{#1}}
\newcommand{\HLJLoB}[1]{\textcolor[RGB]{102,102,102}{\textbf{#1}}}
\newcommand{\HLJLow}[1]{\textcolor[RGB]{102,102,102}{\textbf{#1}}}
\newcommand{\HLJLp}[1]{#1}
\newcommand{\HLJLc}[1]{\textcolor[RGB]{153,153,119}{\textit{#1}}}
\newcommand{\HLJLch}[1]{\textcolor[RGB]{153,153,119}{\textit{#1}}}
\newcommand{\HLJLcm}[1]{\textcolor[RGB]{153,153,119}{\textit{#1}}}
\newcommand{\HLJLcp}[1]{\textcolor[RGB]{153,153,119}{\textit{#1}}}
\newcommand{\HLJLcpB}[1]{\textcolor[RGB]{153,153,119}{\textit{#1}}}
\newcommand{\HLJLcs}[1]{\textcolor[RGB]{153,153,119}{\textit{#1}}}
\newcommand{\HLJLcsB}[1]{\textcolor[RGB]{153,153,119}{\textit{#1}}}
\newcommand{\HLJLg}[1]{#1}
\newcommand{\HLJLgd}[1]{#1}
\newcommand{\HLJLge}[1]{#1}
\newcommand{\HLJLgeB}[1]{#1}
\newcommand{\HLJLgh}[1]{#1}
\newcommand{\HLJLgi}[1]{#1}
\newcommand{\HLJLgo}[1]{#1}
\newcommand{\HLJLgp}[1]{#1}
\newcommand{\HLJLgs}[1]{#1}
\newcommand{\HLJLgsB}[1]{#1}
\newcommand{\HLJLgt}[1]{#1}



\def\qqand{\qquad\hbox{and}\qquad}
\def\qqfor{\qquad\hbox{for}\qquad}
\def\qqas{\qquad\hbox{as}\qquad}
\def\half{ {1 \over 2} }
\def\D{ {\rm d} }
\def\I{ {\rm i} }
\def\E{ {\rm e} }
\def\C{ {\mathbb C} }
\def\R{ {\mathbb R} }
\def\H{ {\mathbb H} }
\def\Z{ {\mathbb Z} }
\def\CC{ {\cal C} }
\def\FF{ {\cal F} }
\def\HH{ {\cal H} }
\def\LL{ {\cal L} }
\def\vc#1{ {\mathbf #1} }
\def\bbC{ {\mathbb C} }



\def\fR{ f_{\rm R} }
\def\fL{ f_{\rm L} }

\def\qqqquad{\qquad\qquad}
\def\qqwhere{\qquad\hbox{where}\qquad}
\def\Res_#1{\underset{#1}{\rm Res}\,}
\def\sech{ {\rm sech}\, }
\def\acos{ {\rm acos}\, }
\def\asin{ {\rm asin}\, }
\def\atan{ {\rm atan}\, }
\def\Ei{ {\rm Ei}\, }
\def\upepsilon{\varepsilon}


\def\Xint#1{ \mathchoice
   {\XXint\displaystyle\textstyle{#1} }%
   {\XXint\textstyle\scriptstyle{#1} }%
   {\XXint\scriptstyle\scriptscriptstyle{#1} }%
   {\XXint\scriptscriptstyle\scriptscriptstyle{#1} }%
   \!\int}
\def\XXint#1#2#3{ {\setbox0=\hbox{$#1{#2#3}{\int}$}
     \vcenter{\hbox{$#2#3$}}\kern-.5\wd0} }
\def\ddashint{\Xint=}
\def\dashint{\Xint-}
% \def\dashint
\def\infdashint{\dashint_{-\infty}^\infty}




\def\addtab#1={#1\;&=}
\def\ccr{\\\addtab}
\def\ip<#1>{\left\langle{#1}\right\rangle}
\def\dx{\D x}
\def\dt{\D t}
\def\dz{\D z}
\def\ds{\D s}

\def\rR{ {\rm R} }
\def\rL{ {\rm L} }

\def\norm#1{\left\| #1 \right\|}

\def\pr(#1){\left({#1}\right)}
\def\br[#1]{\left[{#1}\right]}

\def\abs#1{\left|{#1}\right|}
\def\fpr(#1){\!\pr({#1})}

\def\sopmatrix#1{ \begin{pmatrix}#1\end{pmatrix} }

\def\endash{–}
\def\emdash{—}
\def\mdblksquare{\blacksquare}
\def\lgblksquare{\blacksquare}
\def\scre{\E}
\def\mapengine#1,#2.{\mapfunction{#1}\ifx\void#2\else\mapengine #2.\fi }

\def\map[#1]{\mapengine #1,\void.}

\def\mapenginesep_#1#2,#3.{\mapfunction{#2}\ifx\void#3\else#1\mapengine #3.\fi }

\def\mapsep_#1[#2]{\mapenginesep_{#1}#2,\void.}


\def\vcbr[#1]{\pr(#1)}


\def\bvect[#1,#2]{
{
\def\dots{\cdots}
\def\mapfunction##1{\ | \  ##1}
	\sopmatrix{
		 \,#1\map[#2]\,
	}
}
}



\def\vect[#1]{
{\def\dots{\ldots}
	\vcbr[{#1}]
} }

\def\vectt[#1]{
{\def\dots{\ldots}
	\vect[{#1}]^{\top}
} }

\def\Vectt[#1]{
{
\def\mapfunction##1{##1 \cr}
\def\dots{\vdots}
	\begin{pmatrix}
		\map[#1]
	\end{pmatrix}
} }

\def\addtab#1={#1\;&=}
\def\ccr{\\\addtab}

\def\questionequals{= \!\!\!\!\!\!{\scriptstyle ? \atop }\,\,\,}

\def\Ei{\rm Ei\,}

\begin{document}

\textbf{Applied Complex Analysis (2021)}

\section{Problem sheet 5: Lectures 21-26}
\rule{\textwidth}{1pt}
\subsection{Problem 1}
This problem considers Cauchy and Logarithmic transforms of Laguerre polynomials. Recall from lectures that

\[
\CC_{[0,\infty)}[\E^{-\diamond}](z) = -{\E^{-z} \Ei z \over 2 \pi \I}
\]
for the exponential integral

\[
\Ei z = \int_{-\infty}^z {\E^\zeta \over \zeta} \D \zeta.
\]
\begin{itemize}
\item[1. ] What is

\end{itemize}
\[
	\CC_{[0,\infty)}[\diamond \E^{-\diamond} L_1^{(1)}(\diamond)](z) := {1 \over 2 \pi \I} \int_0^\infty {x \E^{-x} L_1^{(1)}(x) \over x-z} \dx
\]
in terms of $\Ei z$?

\begin{itemize}
\item[2. ] What is

\end{itemize}
\[
	 {1 \over \pi} \int_0^\infty { \E^{-x} L_2(x) \log|z-x|} \dx
\]
in terms of the real and imaginary parts of $\Ei z$?

\rule{\textwidth}{1pt}
\subsection{Problem 2}
Consider the incomplete Gamma function:

\[
\Gamma(\alpha, z) = \int_z^\infty \zeta^{\alpha-1} \E^{-\zeta} \D\zeta,
\]
where the contour of integration is two straight line segments from $z$ to $1$ to $\infty$, hence this has a branch cut on $(-\infty,0]$.

\begin{itemize}
\item[1. ] For $x < 0$ and $\alpha > 0$, show that

\end{itemize}
\[
\Gamma_+(\alpha,x) -  \E^{2\I \pi \alpha} \Gamma_-(\alpha,x) = (1-\E^{2 \I \pi \alpha}) \Gamma(\alpha)
\]
where $\Gamma(\alpha) = \Gamma(\alpha,0) = \int_0^\infty x^{\alpha-1} \E^{-x} \dx$ is the Gamma function and

\[
\Gamma_\pm(\alpha,x)= \lim_{\epsilon \rightarrow0} \Gamma(\alpha, x\pm \I \epsilon).
\]
\begin{itemize}
\item[2. ] For $-1 < \alpha < 0$, express

\end{itemize}
\[
	\CC_{[0,\infty)}[\diamond^\alpha \E^{-\diamond}](z) = {1 \over 2 \pi \I} \int_0^\infty {x^\alpha \E^{-x} \over x-z} \dx
\]
in terms of	  $\Gamma(-\alpha,-z)$ and $(-z)^{\alpha} \E^z$ using Plemelj's lemma.

\rule{\textwidth}{1pt}
\subsection{Problem 3}
Define

\[
a(z) = z^2 - 4 + z^{-2}.
\]
\begin{itemize}
\item[1. ] What are the entries of $L[a(z)]^{-1}$?


\item[2. ] Find the Wiener\ensuremath{\endash}Hopf factorisation

\end{itemize}
\[
a(z) = \phi_+(z) \phi_-(z)
\]
where $\phi_+(z)$ is analytic inside the unit circle and and $\phi_-(z)$ is analytic outside, with $\phi_-(\infty) = 1$.

\begin{itemize}
\item[3. ] Find the UL decomposition

\end{itemize}
\[
T[a(z)] = U L
\]
where $U$ is upper-triangular with $1$ on the diagonal and $L$ is lower triangular.

\begin{itemize}
\item[4. ] What is $T[a(z)]^{-1}$?


\item[5. ] What is $T[(z^2+3)/(z^2+2)]^{-1}$?

\end{itemize}
\rule{\textwidth}{1pt}
\subsection{Problem 4}
When the winding number is non-trivial, a Toeplitz operator can either be non-invertible or have multiple solutions. This problem sheet explores this.

\begin{itemize}
\item[1. ] What is the winding number of $a(z) = z$? Show that

\end{itemize}
\[
T[z] \vc u = \vc f
\]
only has a solution if $f_0$ (the first entry of $\vc f$) is zero.

\begin{itemize}
\item[2. ] What is the winding number of $a(z) = z^{-1}$? What are \emph{all} solutions to

\end{itemize}
\[
T[z^{-1}] \vc u = \vc f?
\]
\begin{itemize}
\item[3. ] Show that if $a(z)$ has winding number $\kappa$ it can be written as

\end{itemize}
\[
a(z) = \phi_+(z) z^{\kappa} \phi_-(z)
\]
What are $\phi_+(z)$ and $\phi_-(z)$ in terms of $\log(a(z) z^{-\kappa})$?

\begin{itemize}
\item[4. ] Show that if the winding number is $\kappa$ there exists a

\end{itemize}
\[
T[a(z)] = UPL
\]
decomposition, where

\[
P = T[z^\kappa]
\]
is a permutation operator.

\begin{itemize}
\item[5. ] Find all solutions to

\end{itemize}
\[
T[1/(2z^2+1)] \vc u = \vc e_0
\]
\rule{\textwidth}{1pt}
\subsection{Problem 5}
This set of problems investigates the analyticity properties of the half-Fourier transform.  Recall the definitions


\begin{align*}
	u_{\rm R}(x) & = \begin{cases} u(x) & x \geq 0 \\
	                             0 & \text{ otherwise}
															 \end{cases}, \\
	u_{\rm L}(x) & = \begin{cases} u(x) & x < 0 \\
	                         0 &\text{ otherwise}
													 \end{cases},
\end{align*}
The Fourier transform

\[
\hat u(s) = \int_{-\infty}^\infty u(x) \E^{-\I x s} \dx,
\]
and the inverse Fourier transform

\[
u(x) = {1 \over 2\pi} \int_{-\infty + \I \gamma}^{\infty + \I \gamma} \hat u(s) \E^{\I s x} \D s
\]
where the choice of $\gamma$ is dictated by the analyticity of $\hat u(z)$.

\begin{itemize}
\item[1. ] Consider $f(x) = x$. Without computing it, in what strip, if any,  is $\hat f(z)$ analytic? For what choice of $\gamma$, if any, does the inverse Fourier transform recover $f$?


\item[2. ] Consider $f(x) = {1 \over 1+ \E^x}$. Without computing it, in what strip, if any,  is $\hat f(z)$ analytic? For what choice of $\gamma$, if any, does the inverse Fourier transform recover $f$?


\item[3. ] Consider $f(x) = \E^{2 x}$.Without computing it, in what strip, if any, is $\widehat{f_{\rm R}}(z)$ analytic? For what choice of $\gamma$, if any, does the inverse Fourier transform recover $f$?


\item[4. ] Consider $f(x) = x$. Without computing it, in what strip, if any, is $\widehat{f_{\rm L}}(z)$ analytic? For what choice of $\gamma$, if any, does the inverse Fourier transform recover $f$?


\item[5. ] Calculate the Fourier transforms in the above problems and confirm your statements.


\item[6. ] What is the Fourier transform of $\delta(x)$, i.e., the Dirac delta function satisfying

\end{itemize}
\[
\int_{-\infty}^\infty \delta(x) g(x) \dx = g(0)
\]
for smooth test functions $g$. Where is it analytic?

\rule{\textwidth}{1pt}
\subsection{Problem 6}
This set of problems considers extensions of the Wiener\ensuremath{\endash}Hopf method to functions that do not decay, degenerate integral equations,  and to integro-differential equations. Please be precise on which contour the Riemann\ensuremath{\endash}Hilbert problem is solved on and the  inverse Fourier transforms  taken.

\begin{itemize}
\item[1. ] The function $u(x)$ is bounded by a polynomial for all $x \geq 0$, including as $x \rightarrow \infty$, and satisfies the integral equation

\end{itemize}
for $x \geq 0$,

\[
      u(x) + {3 \over 2} \int_0^\infty \E^{-|x-t|} u(t)\dt = 1 + \alpha x
\]
where $\alpha$ is a positive constant. Find $u(x)$ for $x \geq 0$. Hint: set up  a Riemann\ensuremath{\endash}Hilbert problem on the contour $\R + \I \gamma$ where $-1 < \gamma < 0$ is arbitrary.

\begin{itemize}
\item[2. ] The function $u(x)$ is bounded by a polynomial for all $x \geq 0$, including as $x \rightarrow \infty$, and satisfies the integral equation

\end{itemize}
for $x \geq 0$,

\[
       \int_0^\infty \E^{-\alpha |x-t|} u(t)\dt = 1 + \alpha x
\]
where $\alpha$ is a positive constant. Find $u(x)$ for $x \geq 0$.  Hint: If you proceed na{\textbackslash}"{\textbackslash}i vely, we arrive at a Riemann\ensuremath{\endash}Hilbert problem of the form

\[
\Phi_+(s) -g(s) \Phi_-(s)  = f(s)     \qqand \lim_{z \rightarrow \infty}\Phi(\infty) = 0
\]
but where $g(\infty) = 0$ instead of $g(\infty) = 1$.  This is not in canonical form, but maybe this example is special. Try writing $\Phi(z) = \kappa(z) Y(z)$ as before but allowing different asymptotic behaviour in $\kappa$ and $Y$ in the different half planes in a way that they cancel out so that $\lim_{z \rightarrow \infty} \Phi(z) = 0$:


\begin{align*}
 \kappa(z) &=  \begin{cases} O(z^{-1}) & \Im z > 0 \\
                             O(z) &  \Im z < 0 \end{cases}  \\
Y(z) &=  \begin{cases}    O(1) & \Im z > 0 \\
                          O(z^{-2}) &  \Im z < 0 \end{cases}.
\end{align*}
\begin{itemize}
\item[3. ] Consider the integral equation

\end{itemize}
\[
       u(t) - \lambda \int_0^\infty \E^{-|x-t|} u(t)\dt =  x
\]
where $0 < \lambda < {1 \over 2}$. Show that, for $x \geq 0$,

\[
	u(x) = A(\sinh \gamma x + \gamma \cosh \gamma x) + {1 \over \gamma^2}\br[x + \pr(\gamma - {1 \over \gamma}) \sinh \gamma x]
\]
where $\gamma^2 = 1 - 2\lambda$ and $A$ is an arbitrary constant.

\rule{\textwidth}{1pt}
\subsection{Problem 7}
A bounded, smooth,  function $u(x)$ satisfies the integro-differential equation

\[
u''(x) - {72 \over 5} \int_0^\infty \E^{-5 |x-t|} u(t) \dt = 1 \qqfor x \geq 0
\]
with $u(0) = 0$.

\begin{itemize}
\item[1. ] Rewrite the integral equation on the half line in the form:

\end{itemize}
\[
u_\rR''(x) - {72 \over 5} \int_{-\infty}^\infty \E^{-5 |x-t|} u(t) \dt = 1_\rR(x) + \alpha \delta(x) + p_\rL(x)
\]
for $\alpha = u'(0)$ and a to-be-specified  $p(x)$. Here $\delta$ is  the Dirac delta function, that is, $\int_{-\infty}^\infty f(x) \delta(x) \dx = f(0).$

\begin{itemize}
\item[2. ] Use integration by parts to determine that

\end{itemize}
\[
\widehat{u_{\rm R}''}(s) = -u'(0) -s^2 \hat u_{\rm R}(s).
\]
What is $\hat{\delta}(s)$? Use these to translate the equation to Fourier space on a contour $s \in \R + \I \gamma$. What choices of $\gamma$ are suitable?

\begin{itemize}
\item[3. ] Define $\Phi(z)$ in  terms of $\widehat{p_\rL}(z)$ and $\widehat{u_\rR}(z)$ so that it satisfies the following (non-standard) RH problem

\end{itemize}

\begin{align*}
\Phi_+(s) - {(s^2 + 9) (s^2+16) \over s^2+25} \Phi_-(s) &= \alpha + {1 \over \I s}\\
 \lim_{z \rightarrow \infty\atop \Im z > \gamma}  \Phi(z) &= \alpha \\
  \lim_{z \rightarrow \infty\atop \Im z < \gamma}  \Phi(z) &= 0.
\end{align*}
\begin{itemize}
\item[4. ] Solve the Riemann\ensuremath{\endash}Hilbert problem for $\Phi$. Hint: write $\Phi(z) = \kappa(z) Y(z)$ where

\end{itemize}

\begin{align*}
 \kappa(z) & =  \begin{cases}  O(z) & \Im z > \gamma \\
                          O(z^{-1}) &  \Im z < \gamma \end{cases},\\
Y(z) & =  O(z^{-1}).
\end{align*}
Hint: $Y(z)$ does not depend on $\alpha$ in the lower-half plane.

\begin{itemize}
\item[5. ] Recover $u(x)$ by taking the inverse Fourier transform of $\Phi_-(s)$.

\end{itemize}
\rule{\textwidth}{1pt}


\end{document}
