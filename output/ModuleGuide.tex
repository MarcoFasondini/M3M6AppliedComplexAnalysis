\documentclass[12pt,a4paper]{article}

\usepackage[a4paper,text={16.5cm,25.2cm},centering]{geometry}
\usepackage{lmodern}
\usepackage{amssymb,amsmath}
\usepackage{bm}
\usepackage{graphicx}
\usepackage{microtype}
\usepackage{hyperref}
\setlength{\parindent}{0pt}
\setlength{\parskip}{1.2ex}

\hypersetup
       {   pdfauthor = { Marco Fasondini },
           pdftitle={ foo },
           colorlinks=TRUE,
           linkcolor=black,
           citecolor=blue,
           urlcolor=blue
       }




\usepackage{upquote}
\usepackage{listings}
\usepackage{xcolor}
\lstset{
    basicstyle=\ttfamily\footnotesize,
    upquote=true,
    breaklines=true,
    breakindent=0pt,
    keepspaces=true,
    showspaces=false,
    columns=fullflexible,
    showtabs=false,
    showstringspaces=false,
    escapeinside={(*@}{@*)},
    extendedchars=true,
}
\newcommand{\HLJLt}[1]{#1}
\newcommand{\HLJLw}[1]{#1}
\newcommand{\HLJLe}[1]{#1}
\newcommand{\HLJLeB}[1]{#1}
\newcommand{\HLJLo}[1]{#1}
\newcommand{\HLJLk}[1]{\textcolor[RGB]{148,91,176}{\textbf{#1}}}
\newcommand{\HLJLkc}[1]{\textcolor[RGB]{59,151,46}{\textit{#1}}}
\newcommand{\HLJLkd}[1]{\textcolor[RGB]{214,102,97}{\textit{#1}}}
\newcommand{\HLJLkn}[1]{\textcolor[RGB]{148,91,176}{\textbf{#1}}}
\newcommand{\HLJLkp}[1]{\textcolor[RGB]{148,91,176}{\textbf{#1}}}
\newcommand{\HLJLkr}[1]{\textcolor[RGB]{148,91,176}{\textbf{#1}}}
\newcommand{\HLJLkt}[1]{\textcolor[RGB]{148,91,176}{\textbf{#1}}}
\newcommand{\HLJLn}[1]{#1}
\newcommand{\HLJLna}[1]{#1}
\newcommand{\HLJLnb}[1]{#1}
\newcommand{\HLJLnbp}[1]{#1}
\newcommand{\HLJLnc}[1]{#1}
\newcommand{\HLJLncB}[1]{#1}
\newcommand{\HLJLnd}[1]{\textcolor[RGB]{214,102,97}{#1}}
\newcommand{\HLJLne}[1]{#1}
\newcommand{\HLJLneB}[1]{#1}
\newcommand{\HLJLnf}[1]{\textcolor[RGB]{66,102,213}{#1}}
\newcommand{\HLJLnfm}[1]{\textcolor[RGB]{66,102,213}{#1}}
\newcommand{\HLJLnp}[1]{#1}
\newcommand{\HLJLnl}[1]{#1}
\newcommand{\HLJLnn}[1]{#1}
\newcommand{\HLJLno}[1]{#1}
\newcommand{\HLJLnt}[1]{#1}
\newcommand{\HLJLnv}[1]{#1}
\newcommand{\HLJLnvc}[1]{#1}
\newcommand{\HLJLnvg}[1]{#1}
\newcommand{\HLJLnvi}[1]{#1}
\newcommand{\HLJLnvm}[1]{#1}
\newcommand{\HLJLl}[1]{#1}
\newcommand{\HLJLld}[1]{\textcolor[RGB]{148,91,176}{\textit{#1}}}
\newcommand{\HLJLs}[1]{\textcolor[RGB]{201,61,57}{#1}}
\newcommand{\HLJLsa}[1]{\textcolor[RGB]{201,61,57}{#1}}
\newcommand{\HLJLsb}[1]{\textcolor[RGB]{201,61,57}{#1}}
\newcommand{\HLJLsc}[1]{\textcolor[RGB]{201,61,57}{#1}}
\newcommand{\HLJLsd}[1]{\textcolor[RGB]{201,61,57}{#1}}
\newcommand{\HLJLsdB}[1]{\textcolor[RGB]{201,61,57}{#1}}
\newcommand{\HLJLsdC}[1]{\textcolor[RGB]{201,61,57}{#1}}
\newcommand{\HLJLse}[1]{\textcolor[RGB]{59,151,46}{#1}}
\newcommand{\HLJLsh}[1]{\textcolor[RGB]{201,61,57}{#1}}
\newcommand{\HLJLsi}[1]{#1}
\newcommand{\HLJLso}[1]{\textcolor[RGB]{201,61,57}{#1}}
\newcommand{\HLJLsr}[1]{\textcolor[RGB]{201,61,57}{#1}}
\newcommand{\HLJLss}[1]{\textcolor[RGB]{201,61,57}{#1}}
\newcommand{\HLJLssB}[1]{\textcolor[RGB]{201,61,57}{#1}}
\newcommand{\HLJLnB}[1]{\textcolor[RGB]{59,151,46}{#1}}
\newcommand{\HLJLnbB}[1]{\textcolor[RGB]{59,151,46}{#1}}
\newcommand{\HLJLnfB}[1]{\textcolor[RGB]{59,151,46}{#1}}
\newcommand{\HLJLnh}[1]{\textcolor[RGB]{59,151,46}{#1}}
\newcommand{\HLJLni}[1]{\textcolor[RGB]{59,151,46}{#1}}
\newcommand{\HLJLnil}[1]{\textcolor[RGB]{59,151,46}{#1}}
\newcommand{\HLJLnoB}[1]{\textcolor[RGB]{59,151,46}{#1}}
\newcommand{\HLJLoB}[1]{\textcolor[RGB]{102,102,102}{\textbf{#1}}}
\newcommand{\HLJLow}[1]{\textcolor[RGB]{102,102,102}{\textbf{#1}}}
\newcommand{\HLJLp}[1]{#1}
\newcommand{\HLJLc}[1]{\textcolor[RGB]{153,153,119}{\textit{#1}}}
\newcommand{\HLJLch}[1]{\textcolor[RGB]{153,153,119}{\textit{#1}}}
\newcommand{\HLJLcm}[1]{\textcolor[RGB]{153,153,119}{\textit{#1}}}
\newcommand{\HLJLcp}[1]{\textcolor[RGB]{153,153,119}{\textit{#1}}}
\newcommand{\HLJLcpB}[1]{\textcolor[RGB]{153,153,119}{\textit{#1}}}
\newcommand{\HLJLcs}[1]{\textcolor[RGB]{153,153,119}{\textit{#1}}}
\newcommand{\HLJLcsB}[1]{\textcolor[RGB]{153,153,119}{\textit{#1}}}
\newcommand{\HLJLg}[1]{#1}
\newcommand{\HLJLgd}[1]{#1}
\newcommand{\HLJLge}[1]{#1}
\newcommand{\HLJLgeB}[1]{#1}
\newcommand{\HLJLgh}[1]{#1}
\newcommand{\HLJLgi}[1]{#1}
\newcommand{\HLJLgo}[1]{#1}
\newcommand{\HLJLgp}[1]{#1}
\newcommand{\HLJLgs}[1]{#1}
\newcommand{\HLJLgsB}[1]{#1}
\newcommand{\HLJLgt}[1]{#1}



\def\qqand{\qquad\hbox{and}\qquad}
\def\qqfor{\qquad\hbox{for}\qquad}
\def\qqas{\qquad\hbox{as}\qquad}
\def\half{ {1 \over 2} }
\def\D{ {\rm d} }
\def\I{ {\rm i} }
\def\E{ {\rm e} }
\def\C{ {\mathbb C} }
\def\R{ {\mathbb R} }
\def\H{ {\mathbb H} }
\def\Z{ {\mathbb Z} }
\def\CC{ {\cal C} }
\def\FF{ {\cal F} }
\def\HH{ {\cal H} }
\def\LL{ {\cal L} }
\def\vc#1{ {\mathbf #1} }
\def\bbC{ {\mathbb C} }



\def\fR{ f_{\rm R} }
\def\fL{ f_{\rm L} }

\def\qqqquad{\qquad\qquad}
\def\qqwhere{\qquad\hbox{where}\qquad}
\def\Res_#1{\underset{#1}{\rm Res}\,}
\def\sech{ {\rm sech}\, }
\def\acos{ {\rm acos}\, }
\def\asin{ {\rm asin}\, }
\def\atan{ {\rm atan}\, }
\def\Ei{ {\rm Ei}\, }
\def\upepsilon{\varepsilon}


\def\Xint#1{ \mathchoice
   {\XXint\displaystyle\textstyle{#1} }%
   {\XXint\textstyle\scriptstyle{#1} }%
   {\XXint\scriptstyle\scriptscriptstyle{#1} }%
   {\XXint\scriptscriptstyle\scriptscriptstyle{#1} }%
   \!\int}
\def\XXint#1#2#3{ {\setbox0=\hbox{$#1{#2#3}{\int}$}
     \vcenter{\hbox{$#2#3$}}\kern-.5\wd0} }
\def\ddashint{\Xint=}
\def\dashint{\Xint-}
% \def\dashint
\def\infdashint{\dashint_{-\infty}^\infty}




\def\addtab#1={#1\;&=}
\def\ccr{\\\addtab}
\def\ip<#1>{\left\langle{#1}\right\rangle}
\def\dx{\D x}
\def\dt{\D t}
\def\dz{\D z}
\def\ds{\D s}

\def\rR{ {\rm R} }
\def\rL{ {\rm L} }

\def\norm#1{\left\| #1 \right\|}

\def\pr(#1){\left({#1}\right)}
\def\br[#1]{\left[{#1}\right]}

\def\abs#1{\left|{#1}\right|}
\def\fpr(#1){\!\pr({#1})}

\def\sopmatrix#1{ \begin{pmatrix}#1\end{pmatrix} }

\def\endash{–}
\def\emdash{—}
\def\mdblksquare{\blacksquare}
\def\lgblksquare{\blacksquare}
\def\scre{\E}
\def\mapengine#1,#2.{\mapfunction{#1}\ifx\void#2\else\mapengine #2.\fi }

\def\map[#1]{\mapengine #1,\void.}

\def\mapenginesep_#1#2,#3.{\mapfunction{#2}\ifx\void#3\else#1\mapengine #3.\fi }

\def\mapsep_#1[#2]{\mapenginesep_{#1}#2,\void.}


\def\vcbr[#1]{\pr(#1)}


\def\bvect[#1,#2]{
{
\def\dots{\cdots}
\def\mapfunction##1{\ | \  ##1}
	\sopmatrix{
		 \,#1\map[#2]\,
	}
}
}



\def\vect[#1]{
{\def\dots{\ldots}
	\vcbr[{#1}]
} }

\def\vectt[#1]{
{\def\dots{\ldots}
	\vect[{#1}]^{\top}
} }

\def\Vectt[#1]{
{
\def\mapfunction##1{##1 \cr}
\def\dots{\vdots}
	\begin{pmatrix}
		\map[#1]
	\end{pmatrix}
} }

\def\addtab#1={#1\;&=}
\def\ccr{\\\addtab}

\def\questionequals{= \!\!\!\!\!\!{\scriptstyle ? \atop }\,\,\,}

\begin{document}

\section{Module Guide: Applied Complex Analysis (Spring Term, 2021)}
\textbf{Module code:} MATH96019/MATH97028/MATH97105

\textbf{Lecturer:} Dr. Marco Fasondini

\textbf{email:} m.fasondini@imperial.ac.uk

\textbf{Website:} On GitHub, at \href{https://github.com/MarcoFasondini/M3M6AppliedComplexAnalysis}{https://github.com/MarcoFasondini/M3M6AppliedComplexAnalysis}

\textbf{Video lectures:}  \href{https://www.imperial.ac.uk/admin-services/ict/self-service/teaching-learning/panopto/panopto-student-user-guide/}{Panopto}

\textbf{Discussion forum:} Sign up \href{https://www.piazza.com/imperial.ac.uk/spring2021/appliedcomplexanalysis2021}{here} to Piazza, where you can ask questions (anonymously, if you wish) and find fellow students to work with. The module's Piazza site is \href{https://www.piazza.com/imperial.ac.uk/spring2021/appliedcomplexanalysis2021/home}{here}

\textbf{Problem classes:} on \href{https://teams.microsoft.com/l/team/19%3a48e62f0e196348b68e070b9e9b4e66f7%40thread.tacv2/conversations?groupId=04eef181-b621-4c5d-a09a-737ac806f2c8&tenantId=2b897507-ee8c-4575-830b-4f8267c3d307}{MS Teams} (for now). There are two problem classes on Thursdays in Weeks 2, 4, 6, 8 and 10; one is from 11:00-12:00, the other from 2:00-3:00. The same problem sheet (available on the module website) will be covered in each of the two problem classes.

\textbf{Office hours:} Fridays, 11:00-12:00, on \href{https://teams.microsoft.com/l/team/19%3a48e62f0e196348b68e070b9e9b4e66f7%40thread.tacv2/conversations?groupId=04eef181-b621-4c5d-a09a-737ac806f2c8&tenantId=2b897507-ee8c-4575-830b-4f8267c3d307}{MS Teams} (for the time being).

\textbf{Reading list:}

\begin{itemize}
\item[1. ] M.J. Ablowitz \& A.S. Fokas, \href{https://www.imperial.ac.uk/admin-services/library/}{Complex Variables: Introduction and Applications}, Second Edition, Cambridge University Press, 2003


\item[2. ] R. Earl, \href{https://courses.maths.ox.ac.uk/node/view_material/5392}{Metric Spaces and Complex Analysis}, 2015


\item[3. ] E. Wegert, \href{https://www.imperial.ac.uk/admin-services/library/}{Visual Complex Functions: An Introduction with Phase Portraits}, Birkhäuser, 2012


\item[4. ] B. Fornberg \& C. Piret, Complex Variables and Analytic Functions: An Illustrated Introduction, SIAM, 2019 (this book might become available through the library later this term)

\end{itemize}
\subsection{Overview of course}
\begin{itemize}
\item[1. ] Complex analysis, Cauchy's theorem, residue calculus


\item[2. ] Matrix-valued functions, with applications to PDEs and fractional-order PDEs


\item[3. ] Singular integrals of the form

\end{itemize}
\[
\int_\Gamma {u(\zeta) \over z - \zeta}  d\zeta,
\]
\[
\int_\Gamma u(\zeta) \log|z - \zeta|  ds
\]
with applications to PDEs, airfoil design, etc.

\begin{itemize}
\item[4. ] Orthogonal polynomials, with applications to Schrödinger operators, solving differential equations.


\item[5. ] Weiner-Hopf method with applications to integral equations with integral operators

\end{itemize}
\[
\int_0^\infty K(x-y) u(y)  dy
\]
\emph{Central themes}:

\begin{itemize}
\item[1. ] Finding "nice" formulae for problems that arise in applications (physics and elsewhere). These can be closed form solution, sums, integral representations, special functions, etc.


\item[2. ] Computational tools for approximate solutions to problems that arise in applications.

\end{itemize}
\emph{Applications} (not necessarily discussed in the course):

\begin{itemize}
\item[1. ] Ideal fluid flow


\item[2. ] Acoustic scattering


\item[3. ] Electrostatics (Faraday cage)


\item[4. ] Fracture mechanics


\item[5. ] Schrödinger equations


\item[6. ] Shallow water waves

\end{itemize}
\subsection{The Project}
There is a project worth 10\% of your final mark (and the exam is worth 90\%). You can decide whether to do the project individually or in a group of up to four students in total. This project is \emph{open ended}: you (or your group) propose a topic. This could be computationally based (possibly based on the slides), theoretically based (possibly looking at material on the reading list), or otherwise. If you are having difficulty coming up with a proposal, please attend the office hours for advice. You can find examples of projects on the module website.

Timeline:

\begin{itemize}
\item 26 Feb: Turn in short description of proposed project (max 2 paragraphs)


\item 26 March: Project due

\end{itemize}


\end{document}
