\documentclass[12pt,a4paper]{article}

\usepackage[a4paper,text={16.5cm,25.2cm},centering]{geometry}
\usepackage{lmodern}
\usepackage{amssymb,amsmath}
\usepackage{bm}
\usepackage{graphicx}
\usepackage{microtype}
\usepackage{hyperref}
\setlength{\parindent}{0pt}
\setlength{\parskip}{1.2ex}

\hypersetup
       {   pdfauthor = { Marco Fasondini },
           pdftitle={ foo },
           colorlinks=TRUE,
           linkcolor=black,
           citecolor=blue,
           urlcolor=blue
       }




\usepackage{upquote}
\usepackage{listings}
\usepackage{xcolor}
\lstset{
    basicstyle=\ttfamily\footnotesize,
    upquote=true,
    breaklines=true,
    breakindent=0pt,
    keepspaces=true,
    showspaces=false,
    columns=fullflexible,
    showtabs=false,
    showstringspaces=false,
    escapeinside={(*@}{@*)},
    extendedchars=true,
}
\newcommand{\HLJLt}[1]{#1}
\newcommand{\HLJLw}[1]{#1}
\newcommand{\HLJLe}[1]{#1}
\newcommand{\HLJLeB}[1]{#1}
\newcommand{\HLJLo}[1]{#1}
\newcommand{\HLJLk}[1]{\textcolor[RGB]{148,91,176}{\textbf{#1}}}
\newcommand{\HLJLkc}[1]{\textcolor[RGB]{59,151,46}{\textit{#1}}}
\newcommand{\HLJLkd}[1]{\textcolor[RGB]{214,102,97}{\textit{#1}}}
\newcommand{\HLJLkn}[1]{\textcolor[RGB]{148,91,176}{\textbf{#1}}}
\newcommand{\HLJLkp}[1]{\textcolor[RGB]{148,91,176}{\textbf{#1}}}
\newcommand{\HLJLkr}[1]{\textcolor[RGB]{148,91,176}{\textbf{#1}}}
\newcommand{\HLJLkt}[1]{\textcolor[RGB]{148,91,176}{\textbf{#1}}}
\newcommand{\HLJLn}[1]{#1}
\newcommand{\HLJLna}[1]{#1}
\newcommand{\HLJLnb}[1]{#1}
\newcommand{\HLJLnbp}[1]{#1}
\newcommand{\HLJLnc}[1]{#1}
\newcommand{\HLJLncB}[1]{#1}
\newcommand{\HLJLnd}[1]{\textcolor[RGB]{214,102,97}{#1}}
\newcommand{\HLJLne}[1]{#1}
\newcommand{\HLJLneB}[1]{#1}
\newcommand{\HLJLnf}[1]{\textcolor[RGB]{66,102,213}{#1}}
\newcommand{\HLJLnfm}[1]{\textcolor[RGB]{66,102,213}{#1}}
\newcommand{\HLJLnp}[1]{#1}
\newcommand{\HLJLnl}[1]{#1}
\newcommand{\HLJLnn}[1]{#1}
\newcommand{\HLJLno}[1]{#1}
\newcommand{\HLJLnt}[1]{#1}
\newcommand{\HLJLnv}[1]{#1}
\newcommand{\HLJLnvc}[1]{#1}
\newcommand{\HLJLnvg}[1]{#1}
\newcommand{\HLJLnvi}[1]{#1}
\newcommand{\HLJLnvm}[1]{#1}
\newcommand{\HLJLl}[1]{#1}
\newcommand{\HLJLld}[1]{\textcolor[RGB]{148,91,176}{\textit{#1}}}
\newcommand{\HLJLs}[1]{\textcolor[RGB]{201,61,57}{#1}}
\newcommand{\HLJLsa}[1]{\textcolor[RGB]{201,61,57}{#1}}
\newcommand{\HLJLsb}[1]{\textcolor[RGB]{201,61,57}{#1}}
\newcommand{\HLJLsc}[1]{\textcolor[RGB]{201,61,57}{#1}}
\newcommand{\HLJLsd}[1]{\textcolor[RGB]{201,61,57}{#1}}
\newcommand{\HLJLsdB}[1]{\textcolor[RGB]{201,61,57}{#1}}
\newcommand{\HLJLsdC}[1]{\textcolor[RGB]{201,61,57}{#1}}
\newcommand{\HLJLse}[1]{\textcolor[RGB]{59,151,46}{#1}}
\newcommand{\HLJLsh}[1]{\textcolor[RGB]{201,61,57}{#1}}
\newcommand{\HLJLsi}[1]{#1}
\newcommand{\HLJLso}[1]{\textcolor[RGB]{201,61,57}{#1}}
\newcommand{\HLJLsr}[1]{\textcolor[RGB]{201,61,57}{#1}}
\newcommand{\HLJLss}[1]{\textcolor[RGB]{201,61,57}{#1}}
\newcommand{\HLJLssB}[1]{\textcolor[RGB]{201,61,57}{#1}}
\newcommand{\HLJLnB}[1]{\textcolor[RGB]{59,151,46}{#1}}
\newcommand{\HLJLnbB}[1]{\textcolor[RGB]{59,151,46}{#1}}
\newcommand{\HLJLnfB}[1]{\textcolor[RGB]{59,151,46}{#1}}
\newcommand{\HLJLnh}[1]{\textcolor[RGB]{59,151,46}{#1}}
\newcommand{\HLJLni}[1]{\textcolor[RGB]{59,151,46}{#1}}
\newcommand{\HLJLnil}[1]{\textcolor[RGB]{59,151,46}{#1}}
\newcommand{\HLJLnoB}[1]{\textcolor[RGB]{59,151,46}{#1}}
\newcommand{\HLJLoB}[1]{\textcolor[RGB]{102,102,102}{\textbf{#1}}}
\newcommand{\HLJLow}[1]{\textcolor[RGB]{102,102,102}{\textbf{#1}}}
\newcommand{\HLJLp}[1]{#1}
\newcommand{\HLJLc}[1]{\textcolor[RGB]{153,153,119}{\textit{#1}}}
\newcommand{\HLJLch}[1]{\textcolor[RGB]{153,153,119}{\textit{#1}}}
\newcommand{\HLJLcm}[1]{\textcolor[RGB]{153,153,119}{\textit{#1}}}
\newcommand{\HLJLcp}[1]{\textcolor[RGB]{153,153,119}{\textit{#1}}}
\newcommand{\HLJLcpB}[1]{\textcolor[RGB]{153,153,119}{\textit{#1}}}
\newcommand{\HLJLcs}[1]{\textcolor[RGB]{153,153,119}{\textit{#1}}}
\newcommand{\HLJLcsB}[1]{\textcolor[RGB]{153,153,119}{\textit{#1}}}
\newcommand{\HLJLg}[1]{#1}
\newcommand{\HLJLgd}[1]{#1}
\newcommand{\HLJLge}[1]{#1}
\newcommand{\HLJLgeB}[1]{#1}
\newcommand{\HLJLgh}[1]{#1}
\newcommand{\HLJLgi}[1]{#1}
\newcommand{\HLJLgo}[1]{#1}
\newcommand{\HLJLgp}[1]{#1}
\newcommand{\HLJLgs}[1]{#1}
\newcommand{\HLJLgsB}[1]{#1}
\newcommand{\HLJLgt}[1]{#1}



\def\qqand{\qquad\hbox{and}\qquad}
\def\qqfor{\qquad\hbox{for}\qquad}
\def\qqas{\qquad\hbox{as}\qquad}
\def\half{ {1 \over 2} }
\def\D{ {\rm d} }
\def\I{ {\rm i} }
\def\E{ {\rm e} }
\def\C{ {\mathbb C} }
\def\R{ {\mathbb R} }
\def\H{ {\mathbb H} }
\def\Z{ {\mathbb Z} }
\def\CC{ {\cal C} }
\def\FF{ {\cal F} }
\def\HH{ {\cal H} }
\def\LL{ {\cal L} }
\def\vc#1{ {\mathbf #1} }
\def\bbC{ {\mathbb C} }



\def\fR{ f_{\rm R} }
\def\fL{ f_{\rm L} }

\def\qqqquad{\qquad\qquad}
\def\qqwhere{\qquad\hbox{where}\qquad}
\def\Res_#1{\underset{#1}{\rm Res}\,}
\def\sech{ {\rm sech}\, }
\def\acos{ {\rm acos}\, }
\def\asin{ {\rm asin}\, }
\def\atan{ {\rm atan}\, }
\def\Ei{ {\rm Ei}\, }
\def\upepsilon{\varepsilon}


\def\Xint#1{ \mathchoice
   {\XXint\displaystyle\textstyle{#1} }%
   {\XXint\textstyle\scriptstyle{#1} }%
   {\XXint\scriptstyle\scriptscriptstyle{#1} }%
   {\XXint\scriptscriptstyle\scriptscriptstyle{#1} }%
   \!\int}
\def\XXint#1#2#3{ {\setbox0=\hbox{$#1{#2#3}{\int}$}
     \vcenter{\hbox{$#2#3$}}\kern-.5\wd0} }
\def\ddashint{\Xint=}
\def\dashint{\Xint-}
% \def\dashint
\def\infdashint{\dashint_{-\infty}^\infty}




\def\addtab#1={#1\;&=}
\def\ccr{\\\addtab}
\def\ip<#1>{\left\langle{#1}\right\rangle}
\def\dx{\D x}
\def\dt{\D t}
\def\dz{\D z}
\def\ds{\D s}

\def\rR{ {\rm R} }
\def\rL{ {\rm L} }

\def\norm#1{\left\| #1 \right\|}

\def\pr(#1){\left({#1}\right)}
\def\br[#1]{\left[{#1}\right]}

\def\abs#1{\left|{#1}\right|}
\def\fpr(#1){\!\pr({#1})}

\def\sopmatrix#1{ \begin{pmatrix}#1\end{pmatrix} }

\def\endash{–}
\def\emdash{—}
\def\mdblksquare{\blacksquare}
\def\lgblksquare{\blacksquare}
\def\scre{\E}
\def\mapengine#1,#2.{\mapfunction{#1}\ifx\void#2\else\mapengine #2.\fi }

\def\map[#1]{\mapengine #1,\void.}

\def\mapenginesep_#1#2,#3.{\mapfunction{#2}\ifx\void#3\else#1\mapengine #3.\fi }

\def\mapsep_#1[#2]{\mapenginesep_{#1}#2,\void.}


\def\vcbr[#1]{\pr(#1)}


\def\bvect[#1,#2]{
{
\def\dots{\cdots}
\def\mapfunction##1{\ | \  ##1}
	\sopmatrix{
		 \,#1\map[#2]\,
	}
}
}



\def\vect[#1]{
{\def\dots{\ldots}
	\vcbr[{#1}]
} }

\def\vectt[#1]{
{\def\dots{\ldots}
	\vect[{#1}]^{\top}
} }

\def\Vectt[#1]{
{
\def\mapfunction##1{##1 \cr}
\def\dots{\vdots}
	\begin{pmatrix}
		\map[#1]
	\end{pmatrix}
} }

\def\addtab#1={#1\;&=}
\def\ccr{\\\addtab}

\def\questionequals{= \!\!\!\!\!\!{\scriptstyle ? \atop }\,\,\,}

\def\Ei{\rm Ei\,}

\begin{document}

\textbf{Applied Complex Analysis (2021)}

\section{Revision questions}
\rule{\textwidth}{1pt}
\subsection{Question 1}
Consider the solution of the following Laplace's equation, using $z = x + \I y$:

\begin{itemize}
\item[1. ] \[
v_{xx} + v_{yy} = 0
\]
for $z \notin [-1,1] \cup \{\pm2\}$,


\item[2. ] \[
v(x,y) = \pm \log |z \mp 2| + O(1)
\]
as $z \rightarrow \pm 2$,


\item[3. ] \[
v(x,y) = o(1)
\]
as  $z \rightarrow \infty$, and


\item[4. ] \[
v(x,0) = \kappa
\]
for $-1 < x < 1$ where $\kappa$ is an unknown constant.

\end{itemize}
This equation models the potential field of two unit charges of opposite sign at $\pm 2$ with a metal sheet that has no net charge placed on $[-1,1]$.

(a) By writing

\[
v(x,y) = \int_{-1}^1 u(t) \log | t - z| \dt + \log|z-2| - \log | z+2|,
\]
show that  the problem of finding $v(x,y)$ can be reformulated as finding $u(x)$ such that

\[
\int_{-1}^1 u(t) \log|t - x| \dt = f(x),
\]
where

\[
 \int_{-1}^1 u(x) \dx = 0.
\]
What is $f(x)$ in this equation? Explain why $v(x,y)$ will thereby satisfy the required four conditions.

(b) Find $u(x)$. Hint: reduce the problem to one of inverting the Hilbert transform.

(c) What is the value of $\kappa$?

\rule{\textwidth}{1pt}
\subsection{Question 2}
The Laguerre polynomials

\[
L_n^{(\alpha)}(x) = {(-1)^n \over n! } x^n + O(x^{n-1}),
\]
where $\alpha > -1$, are orthogonal with respect to

\[
\langle f,g\rangle_{\alpha} = \int_0^{\infty} f(x) g(x) x^{\alpha} \E^{-x}\D x,
\]
and they satisfy

\[
x L_n^{(\alpha)}(x) = - (n+\alpha)L_{n-1}^{(\alpha)}(x) + (2n+\alpha+1) L_n^{(\alpha)}(x) -(n+1)L_{n+1}^{(\alpha)}(x)
\]
and

\[
L_n^{(\alpha)}(x) = L_n^{(\alpha+1)}(x) - L_{n-1}^{(\alpha +1)}(x).
\]
(a) Show that

\[
{\D L_n^{(\alpha)} \over \dx}  = -L_{n-1}^{(\alpha+1)}(x).
\]
(b) Let

\[
\mathbf{L}^{(\alpha)} = \left(
\begin{array}{c}
L_0^{(\alpha)}(x) \\
L_1^{(\alpha)}(x) \\
\vdots
\end{array}
\right).
\]
Give operators $J$, $D$ and $S$ such that

\[
x\mathbf{L}^{(\alpha)} = J\mathbf{L}^{(\alpha)}, \quad
{\D\over \D x}\mathbf{L}^{(\alpha)} = D\mathbf{L}^{(\alpha)}, \quad
\mathbf{L}^{(\alpha)} = S \mathbf{L}^{(\alpha+1)}.
\]
(c) Suppose $u(x)$ has a weighted Laguerre expansion for $x \in [0, \infty)$,

\[
u(x) = \sum_{k = 0}^{\infty}\E^{-x/2}L_k(x)u_k = \E^{-x/2}\mathbf{L}^{\intercal}\mathbf{u}, \qquad
\mathbf{u} = \left(
\begin{array}{c}
u_0 \\
u_1 \\
\vdots
\end{array}
\right),
\]
where we abbreviate $L_k^{(0)}(x)$ as $L_k(x)$ and $\mathbf{L}^{(0)}$ as $\mathbf{L}$. Use the operators $J, D$ and $S$ to represent the ordinary differential operator

\[
u'(x) - x u(x)  \qqfor x \geq 0
\]
as an operator on the coefficients of $u$, where the range of the operator is specified in $\E^{-x/2} \left(\mathbf{L}^{(1)}\right)^{\intercal}$.

\rule{\textwidth}{1pt}
\subsection{Question 3}
Let $u(x)$ solve the integral equation

\[
\int_0^\infty K(t-x) u(t) \dt = f(x) \qquad \hbox{for} \qquad x \geq 0,
\]
where

\[
K(x) = \E^{-|x|}	\qquad \hbox{and}\qquad f(x) = 2 - \E^{-x}.
\]
We will use the notations

\[
g_{\rm L}(x) :=  \begin{cases} g(x) & x< 0 \\ 0 & x \geq 0  \end{cases}, \qquad
g_{\rm R}(x) := \begin{cases}  0 & x < 0 \\ g(x) &x \geq 0 \end{cases},
\]
and the Fourier transform

\[
\hat f(s) := \int_{-\infty}^\infty f(t) \E^{- \I s t} \dt.
\]
(a) What are the regions of analyticity of $\hat K(s)$,  and $\widehat{f_{\rm R}}(s)$? Assuming that  $|u(x)|$ is bounded, what is the region of analyticity of $\widehat{u_{\rm R}}(s)$?  Justify your answers without explicit calculation.

(b) Show that the Fourier transforms satisfy

\[
\hat K(s) = {2 \over 1 + s^2}
\qqand
\widehat{f_{\rm R}}(s)= {2 + \I s \over \I s - s^2}.
\]
(c) For the integral equation above, set up a Riemann\ensuremath{\endash}Hilbert problem of the form

\[
\Phi_+(s) - g(s) \Phi_-(s) = h(s) \qqfor s \in (-\infty, \infty ) + \I \delta,
\]
where $\Phi_+(s)$ is analytic above $(-\infty, \infty ) + \I \delta$, $\Phi_-(s)$ is analytic below $(-\infty, \infty ) + \I \delta$, $\Phi_{\pm}(s)$ decay at infinity, and

\[
g(s) = {2 \over 1 + s^2}.
\]
Explain  the choice of $\delta$ and the definition of $\Phi_\pm(s)$, $g(s)$ and $h(s)$ in terms of the Fourier transforms of $u$, $f$, and $K$.

(d) Is $g(s)$  degenerate? Explain why or why not.

(e) Find a  solution to the homogeneous Riemann\ensuremath{\endash}Hilbert problem

\[
\kappa_+(s) = g(s) \kappa_-(s) \qqfor s \in (-\infty,\infty) + \I \delta
\]
such that $\kappa_+(s)= o(1)$ and $\kappa_-(s) = s + O(1)$ as $s \rightarrow \infty$, where $\delta$ is the same constant as in (c).

(f) Determine $u(x)$.

\rule{\textwidth}{1pt}


\end{document}
