\documentclass[12pt,a4paper]{article}

\usepackage[a4paper,text={16.5cm,25.2cm},centering]{geometry}
\usepackage{lmodern}
\usepackage{amssymb,amsmath}
\usepackage{bm}
\usepackage{graphicx}
\usepackage{microtype}
\usepackage{hyperref}
\setlength{\parindent}{0pt}
\setlength{\parskip}{1.2ex}

\hypersetup
       {   pdfauthor = { Marco Fasondini },
           pdftitle={ foo },
           colorlinks=TRUE,
           linkcolor=black,
           citecolor=blue,
           urlcolor=blue
       }




\usepackage{upquote}
\usepackage{listings}
\usepackage{xcolor}
\lstset{
    basicstyle=\ttfamily\footnotesize,
    upquote=true,
    breaklines=true,
    breakindent=0pt,
    keepspaces=true,
    showspaces=false,
    columns=fullflexible,
    showtabs=false,
    showstringspaces=false,
    escapeinside={(*@}{@*)},
    extendedchars=true,
}
\newcommand{\HLJLt}[1]{#1}
\newcommand{\HLJLw}[1]{#1}
\newcommand{\HLJLe}[1]{#1}
\newcommand{\HLJLeB}[1]{#1}
\newcommand{\HLJLo}[1]{#1}
\newcommand{\HLJLk}[1]{\textcolor[RGB]{148,91,176}{\textbf{#1}}}
\newcommand{\HLJLkc}[1]{\textcolor[RGB]{59,151,46}{\textit{#1}}}
\newcommand{\HLJLkd}[1]{\textcolor[RGB]{214,102,97}{\textit{#1}}}
\newcommand{\HLJLkn}[1]{\textcolor[RGB]{148,91,176}{\textbf{#1}}}
\newcommand{\HLJLkp}[1]{\textcolor[RGB]{148,91,176}{\textbf{#1}}}
\newcommand{\HLJLkr}[1]{\textcolor[RGB]{148,91,176}{\textbf{#1}}}
\newcommand{\HLJLkt}[1]{\textcolor[RGB]{148,91,176}{\textbf{#1}}}
\newcommand{\HLJLn}[1]{#1}
\newcommand{\HLJLna}[1]{#1}
\newcommand{\HLJLnb}[1]{#1}
\newcommand{\HLJLnbp}[1]{#1}
\newcommand{\HLJLnc}[1]{#1}
\newcommand{\HLJLncB}[1]{#1}
\newcommand{\HLJLnd}[1]{\textcolor[RGB]{214,102,97}{#1}}
\newcommand{\HLJLne}[1]{#1}
\newcommand{\HLJLneB}[1]{#1}
\newcommand{\HLJLnf}[1]{\textcolor[RGB]{66,102,213}{#1}}
\newcommand{\HLJLnfm}[1]{\textcolor[RGB]{66,102,213}{#1}}
\newcommand{\HLJLnp}[1]{#1}
\newcommand{\HLJLnl}[1]{#1}
\newcommand{\HLJLnn}[1]{#1}
\newcommand{\HLJLno}[1]{#1}
\newcommand{\HLJLnt}[1]{#1}
\newcommand{\HLJLnv}[1]{#1}
\newcommand{\HLJLnvc}[1]{#1}
\newcommand{\HLJLnvg}[1]{#1}
\newcommand{\HLJLnvi}[1]{#1}
\newcommand{\HLJLnvm}[1]{#1}
\newcommand{\HLJLl}[1]{#1}
\newcommand{\HLJLld}[1]{\textcolor[RGB]{148,91,176}{\textit{#1}}}
\newcommand{\HLJLs}[1]{\textcolor[RGB]{201,61,57}{#1}}
\newcommand{\HLJLsa}[1]{\textcolor[RGB]{201,61,57}{#1}}
\newcommand{\HLJLsb}[1]{\textcolor[RGB]{201,61,57}{#1}}
\newcommand{\HLJLsc}[1]{\textcolor[RGB]{201,61,57}{#1}}
\newcommand{\HLJLsd}[1]{\textcolor[RGB]{201,61,57}{#1}}
\newcommand{\HLJLsdB}[1]{\textcolor[RGB]{201,61,57}{#1}}
\newcommand{\HLJLsdC}[1]{\textcolor[RGB]{201,61,57}{#1}}
\newcommand{\HLJLse}[1]{\textcolor[RGB]{59,151,46}{#1}}
\newcommand{\HLJLsh}[1]{\textcolor[RGB]{201,61,57}{#1}}
\newcommand{\HLJLsi}[1]{#1}
\newcommand{\HLJLso}[1]{\textcolor[RGB]{201,61,57}{#1}}
\newcommand{\HLJLsr}[1]{\textcolor[RGB]{201,61,57}{#1}}
\newcommand{\HLJLss}[1]{\textcolor[RGB]{201,61,57}{#1}}
\newcommand{\HLJLssB}[1]{\textcolor[RGB]{201,61,57}{#1}}
\newcommand{\HLJLnB}[1]{\textcolor[RGB]{59,151,46}{#1}}
\newcommand{\HLJLnbB}[1]{\textcolor[RGB]{59,151,46}{#1}}
\newcommand{\HLJLnfB}[1]{\textcolor[RGB]{59,151,46}{#1}}
\newcommand{\HLJLnh}[1]{\textcolor[RGB]{59,151,46}{#1}}
\newcommand{\HLJLni}[1]{\textcolor[RGB]{59,151,46}{#1}}
\newcommand{\HLJLnil}[1]{\textcolor[RGB]{59,151,46}{#1}}
\newcommand{\HLJLnoB}[1]{\textcolor[RGB]{59,151,46}{#1}}
\newcommand{\HLJLoB}[1]{\textcolor[RGB]{102,102,102}{\textbf{#1}}}
\newcommand{\HLJLow}[1]{\textcolor[RGB]{102,102,102}{\textbf{#1}}}
\newcommand{\HLJLp}[1]{#1}
\newcommand{\HLJLc}[1]{\textcolor[RGB]{153,153,119}{\textit{#1}}}
\newcommand{\HLJLch}[1]{\textcolor[RGB]{153,153,119}{\textit{#1}}}
\newcommand{\HLJLcm}[1]{\textcolor[RGB]{153,153,119}{\textit{#1}}}
\newcommand{\HLJLcp}[1]{\textcolor[RGB]{153,153,119}{\textit{#1}}}
\newcommand{\HLJLcpB}[1]{\textcolor[RGB]{153,153,119}{\textit{#1}}}
\newcommand{\HLJLcs}[1]{\textcolor[RGB]{153,153,119}{\textit{#1}}}
\newcommand{\HLJLcsB}[1]{\textcolor[RGB]{153,153,119}{\textit{#1}}}
\newcommand{\HLJLg}[1]{#1}
\newcommand{\HLJLgd}[1]{#1}
\newcommand{\HLJLge}[1]{#1}
\newcommand{\HLJLgeB}[1]{#1}
\newcommand{\HLJLgh}[1]{#1}
\newcommand{\HLJLgi}[1]{#1}
\newcommand{\HLJLgo}[1]{#1}
\newcommand{\HLJLgp}[1]{#1}
\newcommand{\HLJLgs}[1]{#1}
\newcommand{\HLJLgsB}[1]{#1}
\newcommand{\HLJLgt}[1]{#1}



\def\qqand{\qquad\hbox{and}\qquad}
\def\qqfor{\qquad\hbox{for}\qquad}
\def\qqas{\qquad\hbox{as}\qquad}
\def\half{ {1 \over 2} }
\def\D{ {\rm d} }
\def\I{ {\rm i} }
\def\E{ {\rm e} }
\def\C{ {\mathbb C} }
\def\R{ {\mathbb R} }
\def\H{ {\mathbb H} }
\def\Z{ {\mathbb Z} }
\def\CC{ {\cal C} }
\def\FF{ {\cal F} }
\def\HH{ {\cal H} }
\def\LL{ {\cal L} }
\def\vc#1{ {\mathbf #1} }
\def\bbC{ {\mathbb C} }



\def\fR{ f_{\rm R} }
\def\fL{ f_{\rm L} }

\def\qqqquad{\qquad\qquad}
\def\qqwhere{\qquad\hbox{where}\qquad}
\def\Res_#1{\underset{#1}{\rm Res}\,}
\def\sech{ {\rm sech}\, }
\def\acos{ {\rm acos}\, }
\def\asin{ {\rm asin}\, }
\def\atan{ {\rm atan}\, }
\def\Ei{ {\rm Ei}\, }
\def\upepsilon{\varepsilon}


\def\Xint#1{ \mathchoice
   {\XXint\displaystyle\textstyle{#1} }%
   {\XXint\textstyle\scriptstyle{#1} }%
   {\XXint\scriptstyle\scriptscriptstyle{#1} }%
   {\XXint\scriptscriptstyle\scriptscriptstyle{#1} }%
   \!\int}
\def\XXint#1#2#3{ {\setbox0=\hbox{$#1{#2#3}{\int}$}
     \vcenter{\hbox{$#2#3$}}\kern-.5\wd0} }
\def\ddashint{\Xint=}
\def\dashint{\Xint-}
% \def\dashint
\def\infdashint{\dashint_{-\infty}^\infty}




\def\addtab#1={#1\;&=}
\def\ccr{\\\addtab}
\def\ip<#1>{\left\langle{#1}\right\rangle}
\def\dx{\D x}
\def\dt{\D t}
\def\dz{\D z}
\def\ds{\D s}

\def\rR{ {\rm R} }
\def\rL{ {\rm L} }

\def\norm#1{\left\| #1 \right\|}

\def\pr(#1){\left({#1}\right)}
\def\br[#1]{\left[{#1}\right]}

\def\abs#1{\left|{#1}\right|}
\def\fpr(#1){\!\pr({#1})}

\def\sopmatrix#1{ \begin{pmatrix}#1\end{pmatrix} }

\def\endash{–}
\def\emdash{—}
\def\mdblksquare{\blacksquare}
\def\lgblksquare{\blacksquare}
\def\scre{\E}
\def\mapengine#1,#2.{\mapfunction{#1}\ifx\void#2\else\mapengine #2.\fi }

\def\map[#1]{\mapengine #1,\void.}

\def\mapenginesep_#1#2,#3.{\mapfunction{#2}\ifx\void#3\else#1\mapengine #3.\fi }

\def\mapsep_#1[#2]{\mapenginesep_{#1}#2,\void.}


\def\vcbr[#1]{\pr(#1)}


\def\bvect[#1,#2]{
{
\def\dots{\cdots}
\def\mapfunction##1{\ | \  ##1}
	\sopmatrix{
		 \,#1\map[#2]\,
	}
}
}



\def\vect[#1]{
{\def\dots{\ldots}
	\vcbr[{#1}]
} }

\def\vectt[#1]{
{\def\dots{\ldots}
	\vect[{#1}]^{\top}
} }

\def\Vectt[#1]{
{
\def\mapfunction##1{##1 \cr}
\def\dots{\vdots}
	\begin{pmatrix}
		\map[#1]
	\end{pmatrix}
} }

\def\addtab#1={#1\;&=}
\def\ccr{\\\addtab}

\def\questionequals{= \!\!\!\!\!\!{\scriptstyle ? \atop }\,\,\,}

\begin{document}

\textbf{Applied Complex Analysis (2021)}

\section{Problem sheet 3}
\rule{\textwidth}{1pt}
\subsection{Problem 1.1}
Use the Plemelj formulae to calculate the following:

\begin{itemize}
\item[1. ] \[
{1 \over 2 \pi \I} \int_{-1}^1 {\sqrt{1-t^2} \over (1+t^2)(t -z)} \dt
\]
for $z \notin [-1,1]$.


\item[2. ] \[
{1 \over 2 \pi \I} \int_{-1}^1 {1 \over (t -z) (2+t)} \dt
\]
for $z \notin [-1,1]$.


\item[3. ] \[
\dashint_{-1}^1 {t \over (t -x) \sqrt{1-t^2} } \dt
\]
for $-1 < x < 1$.

\end{itemize}
\subsection{Problem 1.2}
Find all solutions $\phi(z)$ analytic on $\bar \C \backslash [-1,1]$ with weaker than pole singularities satisfying the following, where $-1 < x <1$:

\begin{itemize}
\item[1. ] \[
\phi_+(x) + \phi_-(x) = 1
\]
and $\phi(\infty) = 0$


\item[2. ] \[
\phi_+(x) + \phi_-(x) = 0
\]
and $\phi(\infty) = 1$


\item[3. ] \[
\phi_+(x) + \phi_-(x) = \sqrt{1-x^2}
\]
and $\phi(\infty) = 0$


\item[4. ] \[
\phi_+(x) + \phi_-(x) = {1 \over x^2 + 1}
\]
$\phi(\infty) = 0$ and $\lim_{z \rightarrow \infty} z \phi(z) = 0$.

\end{itemize}
\subsection{Problem 1.3}
Use Plemelj formulae to find all solutions $u(x)$ defined on $[-1,1]$ to the following, where $-1 < x < 1$:

\begin{itemize}
\item[1. ] \[
{1 \over \pi} \dashint_{-1}^1 {u(t) \over t -x} \dt = {x \over \sqrt{1-x^2}}
\]
.


\item[2. ] \[
{1 \over \pi} \dashint_{-1}^1 {u(t) \over t -x} \dt = {1 \over 2 + x}
\]
where $u$ is bounded at the right-endpoint.

\end{itemize}
\rule{\textwidth}{1pt}
In the following problems, use only the definitions


\begin{align*}
	\log z  &= \int_1^z {1 \over \zeta} \D\zeta \qqfor z \notin (-\infty,0] \\
		\log_\pm x  &= \log(x \pm \I \epsilon)\qqfor x \in (-\infty,0]
\end{align*}
For example, do not use $\log z = \log |z| + \I \arg z$ as we need to prove it first!  You can use the result from lectures  that $\log z^{-1} = - \log z$ and $\log_\pm x =  \log |x| \pm \I \pi$.

\subsection{Problem 2.1}
Show that $\log(ab) = \log a + \log b$ provided that the closed contour defined by the oriented line segments $\gamma = [1, 1/b] \cup [1/b, a] \cup [a, 1]$ does not surround the origin.  Show that if $\gamma$ surrounds the origin counter-clockwise then

\[
	\log(ab) = \log a + \log b + 2 \pi \I
\]
and if $\gamma$ surrounds the origin  clockwise then

\[
	\log(ab) = \log a + \log b - 2 \pi \I
\]
Use $\log_\pm x$ to express the equivalent formulae for all cases where $\gamma$ passes through the origin.

\subsection{Problem 2.2}
Show that $\overline{\log z} = \log \bar z $. Use this to show that $\log z = \log |z| + \I \arg z$. (Hint: deform along an arc for the imaginary part.)

\subsection{Problem 2.3}
Consider $\log_1 z$ defined off $[0,\infty)$ by

\[
\log_1  z := \begin{cases}
	\log z & \text{if } \Im z > 0 \\
	\log_+ z & \text{if } \Im z = 0 \text{ and } z <  0	 \\
	\log z + 2 \pi \I & \text{if } \Im z < 0
	\end{cases}
\]
show that $\log_1 z$ is analytic in $\C \backslash [0,\infty)$ and for $x > 0$

\[
	\lim_{\epsilon \rightarrow 0} \log_1 (x- \I \epsilon) = \lim_{\epsilon \rightarrow 0} \log_1 (x+ \I \epsilon) + 2 \pi \I.
\]
(This is the analytic continuation of $\log z$ over its branch cut.)

\rule{\textwidth}{1pt}
Consider the Cauchy transform over the unit circle ${\Bbb T} = \{ z : |z| = 1\}$:

\[
\CC_{\Bbb T} f(z) = {1 \over 2 \pi \I} \oint {f(\zeta) \over \zeta - z} \D \zeta
\]
Denote the limit from the left/right (inside/outside) as


\begin{align*}
\CC_{\Bbb T}^+f (\zeta) & = \lim_{\epsilon \rightarrow 0} \CC_{\Bbb T} f((1-\epsilon) \zeta) \\
\CC_{\Bbb T}^-f (\zeta) & = \lim_{\epsilon \rightarrow 0} \CC_{\Bbb T} f((1+\epsilon) \zeta)
\end{align*}
\subsection{Problem 3.1}
Assuming that $f$ is analytic in an annulus containing ${\Bbb T}$, show that $\CC_{\Bbb T} f(z)$ satisfies the following (Plemelj formulae on the circle):

\begin{itemize}
\item[1. ] \[
\CC_{\Bbb T} f(z)
\]
is analytic in $\bar\C \backslash {\Bbb T}$


\item[2. ] \[
\CC_{\Bbb T}^+ f(\zeta) - \CC_{\Bbb T}^- f(\zeta) = f(\zeta)
\]

\item[3. ] \[
\CC_{\Bbb T} f(\infty) = 0
\]
.

\end{itemize}
\subsection{Problem 3.2}
Show that it is the unique function $\phi(z)$ satisfying

\begin{itemize}
\item[1. ] \[
\phi(z)
\]
is analytic in $\bar\C \backslash {\Bbb T}$


\item[2. ] \[
\phi^+(\zeta) -\phi^-(\zeta) = f(\zeta)
\]
where $f$ is analytic in an annulus containing ${\Bbb T}$,


\item[3. ] \[
\phi(\infty) = 0
\]
.

\end{itemize}
where


\begin{align*}
\phi^+(\zeta) & = \lim_{\epsilon \rightarrow 0} \phi((1-\epsilon) \zeta), \\
\phi^-(\zeta) & = \lim_{\epsilon \rightarrow 0} \phi((1+\epsilon) \zeta)
\end{align*}
You can assume that $\phi^\pm(\zeta)$ converges uniformly.

\subsection{Problem 3.3}
What is $\CC_{\Bbb T}[\diamond^k](z)$?  What about  $\Re \CC_{\Bbb T}^- [\diamond^k](\zeta)$ and $\Im \CC_{\Bbb T}^- [\diamond^k](\zeta)$? Express your answers separately for negative and non-positive $k$ and justify the answers using 3.1 and 3.2.

\subsection{Problem 3.4}
Construct the solution to ideal fluid flow around a circle, that is, to find a function $v(x,y)$ satisfying

\begin{itemize}
\item[1. ] \[
v_{xx} + v_{yy} = 0
\]
for $x^2 + y^2 > 1$.


\item[2. ] \[
v(x,y) \sim y \cos \theta  - x \sin \theta
\]

\item[3. ] \[
v(x,y) = 0
\]
for $x^2 + y^2 = 1$.

\end{itemize}
\rule{\textwidth}{1pt}
We investigate the solutions to  $\phi^+(x) - c \phi^-(x) = f(x)$ on $[-1,1]$.

\subsection{Problem 4.1}
Describe all solutions $\kappa(z)$ to

\begin{itemize}
\item[1. ] \[
\kappa(z)
\]
is analytic off $[-1,1]$.


\item[2. ] \[
\kappa(\infty) = 0
\]
.


\item[3. ] \[
\kappa
\]
has weaker than pole singularities at $\pm 1$.


\item[4. ] $\kappa^+(x) - \E^{\I \theta} \kappa^-(x) = 0$ for $-1 < x <1$.

\end{itemize}
Show that your answer satisfies all three properties.

\subsection{Problem 4.2}
Construct all solutions $\phi(z)$  to

\begin{itemize}
\item[1. ] \[
\phi(z)
\]
is analytic off $[-1,1]$.


\item[2. ] \[
\phi(\infty) = 0
\]
.


\item[3. ] \[
\phi
\]
has weaker than pole singularities at $\pm 1$.


\item[4. ] \[
\phi^+(x) - \E^{\I \theta} \phi^-(x) = f(x)
\]
.

\end{itemize}
in terms of a Cauchy transform involving $f$. You can assume that $f$ is smooth (infinitely-differentiable on $[-1,1]$), and use the fact that

\[
\CC_{[-1,1]} [(1-\diamond)^\alpha (1+\diamond)^\beta f(\diamond) ](z)
\]
is bounded for all $z$ if $\alpha, \beta > 0$ when $f$ is smooth.

\subsection{Problem 4.3}
Let $c \in \C$. Repeat Problem 4.1 and Problem 4.2 for $\phi^+(x) - c \phi^-(x) = f(x)$.

\rule{\textwidth}{1pt}
We investigate the solutions to  $\phi^+(x) + \phi^-(x) = f(x)$ on two intervals $(-1,-a) \cup (a, 1)$, where $0 < a < 1$.

\subsection{Problem 5.1}
Show that $\kappa(z) = {1 \over \sqrt{z-1} \sqrt{z-a} \sqrt{z+a} \sqrt{z+1}}$ satisfies the following properties:

\begin{itemize}
\item[1. ] \[
\kappa(z)
\]
is analytic off $[-1,-a] \cup [1,a]$.


\item[2. ] \[
\kappa^+(x) + \kappa^-(x) = 0
\]
for $a < |x| < 1$.


\item[3. ] \[
\kappa
\]
has weaker than pole singularities everywhere.


\item[4. ] \[
\kappa(\infty) = 0
\]
.

\end{itemize}
\subsection{Problem 5.2}
Describe all solutions $\psi(z)$ satisfying:

\begin{itemize}
\item[1. ] \[
\psi(z)
\]
is analytic off $[-1,-a] \cup [1,a]$.


\item[2. ] \[
\psi^+(x) + \psi^-(x) = 0
\]
for $a < |x| < 1$.


\item[3. ] \[
\psi
\]
has weaker than pole singularities everywhere.


\item[4. ] \[
\psi(\infty) = 0
\]
.

\end{itemize}
\subsection{Problem 5.3}
Assuming $f(x)$ is smooth (infinitely differentiable on $[-1,-a] \cup [a, 1]$), express  in terms of a Cauchy transform involving $f$ all solutions $\phi(z)$ to the following:

\begin{itemize}
\item[1. ] \[
\phi(z)
\]
is analytic off $[-1,-a] \cup [1,a]$.


\item[2. ] \[
\phi^+(x) + \phi^-(x) = f(x)
\]
for $a < |x| < 1$.


\item[3. ] \[
\phi
\]
has weaker than pole singularities everywhere.


\item[4. ] \[
\phi(\infty) = 0
\]
.

\end{itemize}
\rule{\textwidth}{1pt}
\subsection{Problem 6.1}
Describe the limiting distribution of electric charges under the potential $V(x) = x^4$, that is, the limit of

\[
{\D^2 \lambda_k \over \D t^2} + {\D \lambda_k \over \D t}  = \sum_{j=1 \atop j \neq k}^N {1 \over \lambda_k -\lambda_j} - V'(\lambda_k)
\]
for $k = 1,\ldots, N$ as $N \rightarrow \infty$.  (Hint: scale $\lambda_k$ appropriately.)

\rule{\textwidth}{1pt}


\end{document}
