\documentclass[12pt,a4paper]{article}

\usepackage[a4paper,text={16.5cm,25.2cm},centering]{geometry}
\usepackage{lmodern}
\usepackage{amssymb,amsmath}
\usepackage{bm}
\usepackage{graphicx}
\usepackage{microtype}
\usepackage{hyperref}
\setlength{\parindent}{0pt}
\setlength{\parskip}{1.2ex}

\hypersetup
       {   pdfauthor = { Marco Fasondini },
           pdftitle={ foo },
           colorlinks=TRUE,
           linkcolor=black,
           citecolor=blue,
           urlcolor=blue
       }




\usepackage{upquote}
\usepackage{listings}
\usepackage{xcolor}
\lstset{
    basicstyle=\ttfamily\footnotesize,
    upquote=true,
    breaklines=true,
    breakindent=0pt,
    keepspaces=true,
    showspaces=false,
    columns=fullflexible,
    showtabs=false,
    showstringspaces=false,
    escapeinside={(*@}{@*)},
    extendedchars=true,
}
\newcommand{\HLJLt}[1]{#1}
\newcommand{\HLJLw}[1]{#1}
\newcommand{\HLJLe}[1]{#1}
\newcommand{\HLJLeB}[1]{#1}
\newcommand{\HLJLo}[1]{#1}
\newcommand{\HLJLk}[1]{\textcolor[RGB]{148,91,176}{\textbf{#1}}}
\newcommand{\HLJLkc}[1]{\textcolor[RGB]{59,151,46}{\textit{#1}}}
\newcommand{\HLJLkd}[1]{\textcolor[RGB]{214,102,97}{\textit{#1}}}
\newcommand{\HLJLkn}[1]{\textcolor[RGB]{148,91,176}{\textbf{#1}}}
\newcommand{\HLJLkp}[1]{\textcolor[RGB]{148,91,176}{\textbf{#1}}}
\newcommand{\HLJLkr}[1]{\textcolor[RGB]{148,91,176}{\textbf{#1}}}
\newcommand{\HLJLkt}[1]{\textcolor[RGB]{148,91,176}{\textbf{#1}}}
\newcommand{\HLJLn}[1]{#1}
\newcommand{\HLJLna}[1]{#1}
\newcommand{\HLJLnb}[1]{#1}
\newcommand{\HLJLnbp}[1]{#1}
\newcommand{\HLJLnc}[1]{#1}
\newcommand{\HLJLncB}[1]{#1}
\newcommand{\HLJLnd}[1]{\textcolor[RGB]{214,102,97}{#1}}
\newcommand{\HLJLne}[1]{#1}
\newcommand{\HLJLneB}[1]{#1}
\newcommand{\HLJLnf}[1]{\textcolor[RGB]{66,102,213}{#1}}
\newcommand{\HLJLnfm}[1]{\textcolor[RGB]{66,102,213}{#1}}
\newcommand{\HLJLnp}[1]{#1}
\newcommand{\HLJLnl}[1]{#1}
\newcommand{\HLJLnn}[1]{#1}
\newcommand{\HLJLno}[1]{#1}
\newcommand{\HLJLnt}[1]{#1}
\newcommand{\HLJLnv}[1]{#1}
\newcommand{\HLJLnvc}[1]{#1}
\newcommand{\HLJLnvg}[1]{#1}
\newcommand{\HLJLnvi}[1]{#1}
\newcommand{\HLJLnvm}[1]{#1}
\newcommand{\HLJLl}[1]{#1}
\newcommand{\HLJLld}[1]{\textcolor[RGB]{148,91,176}{\textit{#1}}}
\newcommand{\HLJLs}[1]{\textcolor[RGB]{201,61,57}{#1}}
\newcommand{\HLJLsa}[1]{\textcolor[RGB]{201,61,57}{#1}}
\newcommand{\HLJLsb}[1]{\textcolor[RGB]{201,61,57}{#1}}
\newcommand{\HLJLsc}[1]{\textcolor[RGB]{201,61,57}{#1}}
\newcommand{\HLJLsd}[1]{\textcolor[RGB]{201,61,57}{#1}}
\newcommand{\HLJLsdB}[1]{\textcolor[RGB]{201,61,57}{#1}}
\newcommand{\HLJLsdC}[1]{\textcolor[RGB]{201,61,57}{#1}}
\newcommand{\HLJLse}[1]{\textcolor[RGB]{59,151,46}{#1}}
\newcommand{\HLJLsh}[1]{\textcolor[RGB]{201,61,57}{#1}}
\newcommand{\HLJLsi}[1]{#1}
\newcommand{\HLJLso}[1]{\textcolor[RGB]{201,61,57}{#1}}
\newcommand{\HLJLsr}[1]{\textcolor[RGB]{201,61,57}{#1}}
\newcommand{\HLJLss}[1]{\textcolor[RGB]{201,61,57}{#1}}
\newcommand{\HLJLssB}[1]{\textcolor[RGB]{201,61,57}{#1}}
\newcommand{\HLJLnB}[1]{\textcolor[RGB]{59,151,46}{#1}}
\newcommand{\HLJLnbB}[1]{\textcolor[RGB]{59,151,46}{#1}}
\newcommand{\HLJLnfB}[1]{\textcolor[RGB]{59,151,46}{#1}}
\newcommand{\HLJLnh}[1]{\textcolor[RGB]{59,151,46}{#1}}
\newcommand{\HLJLni}[1]{\textcolor[RGB]{59,151,46}{#1}}
\newcommand{\HLJLnil}[1]{\textcolor[RGB]{59,151,46}{#1}}
\newcommand{\HLJLnoB}[1]{\textcolor[RGB]{59,151,46}{#1}}
\newcommand{\HLJLoB}[1]{\textcolor[RGB]{102,102,102}{\textbf{#1}}}
\newcommand{\HLJLow}[1]{\textcolor[RGB]{102,102,102}{\textbf{#1}}}
\newcommand{\HLJLp}[1]{#1}
\newcommand{\HLJLc}[1]{\textcolor[RGB]{153,153,119}{\textit{#1}}}
\newcommand{\HLJLch}[1]{\textcolor[RGB]{153,153,119}{\textit{#1}}}
\newcommand{\HLJLcm}[1]{\textcolor[RGB]{153,153,119}{\textit{#1}}}
\newcommand{\HLJLcp}[1]{\textcolor[RGB]{153,153,119}{\textit{#1}}}
\newcommand{\HLJLcpB}[1]{\textcolor[RGB]{153,153,119}{\textit{#1}}}
\newcommand{\HLJLcs}[1]{\textcolor[RGB]{153,153,119}{\textit{#1}}}
\newcommand{\HLJLcsB}[1]{\textcolor[RGB]{153,153,119}{\textit{#1}}}
\newcommand{\HLJLg}[1]{#1}
\newcommand{\HLJLgd}[1]{#1}
\newcommand{\HLJLge}[1]{#1}
\newcommand{\HLJLgeB}[1]{#1}
\newcommand{\HLJLgh}[1]{#1}
\newcommand{\HLJLgi}[1]{#1}
\newcommand{\HLJLgo}[1]{#1}
\newcommand{\HLJLgp}[1]{#1}
\newcommand{\HLJLgs}[1]{#1}
\newcommand{\HLJLgsB}[1]{#1}
\newcommand{\HLJLgt}[1]{#1}



\def\qqand{\qquad\hbox{and}\qquad}
\def\qqfor{\qquad\hbox{for}\qquad}
\def\qqas{\qquad\hbox{as}\qquad}
\def\half{ {1 \over 2} }
\def\D{ {\rm d} }
\def\I{ {\rm i} }
\def\E{ {\rm e} }
\def\C{ {\mathbb C} }
\def\R{ {\mathbb R} }
\def\H{ {\mathbb H} }
\def\Z{ {\mathbb Z} }
\def\CC{ {\cal C} }
\def\FF{ {\cal F} }
\def\HH{ {\cal H} }
\def\LL{ {\cal L} }
\def\vc#1{ {\mathbf #1} }
\def\bbC{ {\mathbb C} }



\def\fR{ f_{\rm R} }
\def\fL{ f_{\rm L} }

\def\qqqquad{\qquad\qquad}
\def\qqwhere{\qquad\hbox{where}\qquad}
\def\Res_#1{\underset{#1}{\rm Res}\,}
\def\sech{ {\rm sech}\, }
\def\acos{ {\rm acos}\, }
\def\asin{ {\rm asin}\, }
\def\atan{ {\rm atan}\, }
\def\Ei{ {\rm Ei}\, }
\def\upepsilon{\varepsilon}


\def\Xint#1{ \mathchoice
   {\XXint\displaystyle\textstyle{#1} }%
   {\XXint\textstyle\scriptstyle{#1} }%
   {\XXint\scriptstyle\scriptscriptstyle{#1} }%
   {\XXint\scriptscriptstyle\scriptscriptstyle{#1} }%
   \!\int}
\def\XXint#1#2#3{ {\setbox0=\hbox{$#1{#2#3}{\int}$}
     \vcenter{\hbox{$#2#3$}}\kern-.5\wd0} }
\def\ddashint{\Xint=}
\def\dashint{\Xint-}
% \def\dashint
\def\infdashint{\dashint_{-\infty}^\infty}




\def\addtab#1={#1\;&=}
\def\ccr{\\\addtab}
\def\ip<#1>{\left\langle{#1}\right\rangle}
\def\dx{\D x}
\def\dt{\D t}
\def\dz{\D z}
\def\ds{\D s}

\def\rR{ {\rm R} }
\def\rL{ {\rm L} }

\def\norm#1{\left\| #1 \right\|}

\def\pr(#1){\left({#1}\right)}
\def\br[#1]{\left[{#1}\right]}

\def\abs#1{\left|{#1}\right|}
\def\fpr(#1){\!\pr({#1})}

\def\sopmatrix#1{ \begin{pmatrix}#1\end{pmatrix} }

\def\endash{–}
\def\emdash{—}
\def\mdblksquare{\blacksquare}
\def\lgblksquare{\blacksquare}
\def\scre{\E}
\def\mapengine#1,#2.{\mapfunction{#1}\ifx\void#2\else\mapengine #2.\fi }

\def\map[#1]{\mapengine #1,\void.}

\def\mapenginesep_#1#2,#3.{\mapfunction{#2}\ifx\void#3\else#1\mapengine #3.\fi }

\def\mapsep_#1[#2]{\mapenginesep_{#1}#2,\void.}


\def\vcbr[#1]{\pr(#1)}


\def\bvect[#1,#2]{
{
\def\dots{\cdots}
\def\mapfunction##1{\ | \  ##1}
	\sopmatrix{
		 \,#1\map[#2]\,
	}
}
}



\def\vect[#1]{
{\def\dots{\ldots}
	\vcbr[{#1}]
} }

\def\vectt[#1]{
{\def\dots{\ldots}
	\vect[{#1}]^{\top}
} }

\def\Vectt[#1]{
{
\def\mapfunction##1{##1 \cr}
\def\dots{\vdots}
	\begin{pmatrix}
		\map[#1]
	\end{pmatrix}
} }

\def\addtab#1={#1\;&=}
\def\ccr{\\\addtab}

\def\questionequals{= \!\!\!\!\!\!{\scriptstyle ? \atop }\,\,\,}

\begin{document}

\textbf{Applied Complex Analysis (2021)}

\section{Problem sheet 4: Lectures 16-20}
\rule{\textwidth}{1pt}
\subsection{Problem 1}
\begin{itemize}
\item[1. ] Calculate

\end{itemize}
\[
\int_{-1}^1 \log|x - z| x \dx.
\]
\begin{itemize}
\item[2. ] Calculate

\end{itemize}
\[
\int_{-1}^1 \log|x - z| \sqrt{1-x^2} \dx.
\]
Hint: Use

\[
{\D \over \dx}\br[x \sqrt{1-x^2} + \asin x] = 2 \sqrt{1-x^2}
\]
and the fact that

\[
{\pi \over 2} - \asin x = \acos x.
\]
\begin{itemize}
\item[3. ] Solve the logarithmic singular integral equation:

\end{itemize}
\[
\int_{-1}^1 \log|x-t| u(t)  \dt = {1 \over x^2 + 1}.
\]
You may express your solution in terms of the constant

\[
C = \int_{-1}^1 \log | t| u(t) \dt.
\]
\rule{\textwidth}{1pt}
\subsection{Problem 2}
Consider the problem of the potential field generated by a metal sheet on $[-1,1]$ with a point source with positive unit charge located at $(x,y) = (0,1)$, or in complex coordinates $z = x + \I y$, at $z = \I$.

\begin{itemize}
\item[1. ] Express the problem as a solution $v(x,y)$ to Laplace's equation off $[-1,1]$. You can assume that the metal sheet has no net charge, so that the field at infinity is given by $v(x,y) = \log\abs{z - \I} + o(1)$ where $z = x + \I y$.


\item[2. ] Reduce the Laplace's equation to a singular integral equation of the form:

\end{itemize}
\[
\int_{-1}^1 u(t) \log |x-t| \dt = f(t)
\]
where $u$ is a new unknown. What is $f(t)$ and what is the relationship between $v$ and $\int_{-1}^1 u(x) \log|z -x| \dx$?

\begin{itemize}
\item[3. ] Solve the singular integral equation for $u$.


\item[4. ] What is $v(x,y)$?

\end{itemize}
\rule{\textwidth}{1pt}
\subsection{Problem 3}
This problem set considers the Laguerre polynomials

\[
L_n^{(\alpha)}(x) = {(-1)^n \over n! } x^n + O(x^{n-1})
\]
where $\alpha > -1$, which are orthogonal with respect to

\[
\ip<f,g>_\alpha = \int_0^\infty f(x) g(x) x^\alpha \E^{-x}\dx
\]
We also use the notation $L_n(x) = L_n^{(0)}(x)$.

\begin{itemize}
\item[1. ] Show that the Rodrigues formula holds:

\end{itemize}
\[
L_n^{(\alpha)}(x) = {x^{-\alpha} \E^x \over n!} {\D^n \over \dx^n}\br[x^{\alpha+n}\E^{-x}].
\]
\begin{itemize}
\item[2. ] Show that the derivatives form a hierarchy: we have

\end{itemize}

\begin{align*}
{\D L_n^{(\alpha)} \over \dx} & = -L_{n-1}^{(\alpha+1)}(x), \\
{\D \over \dx}\br[x^{\alpha+1} \E^{-x} L_n^{(\alpha+1)}(x) ]  &= (n+1)x^{\alpha} \E^{-x}L_{n+1}^{(\alpha)}(x), \\
x L_n^{(\alpha+1)}(x) &= -(n+1)L_{n+1}^{(\alpha)}(x) + (n+\alpha+1)L_n^{(\alpha)}(x), \\
L_n^{(\alpha)}(x) &= L_n^{(\alpha+1)}(x) - L_{n-1}^{(\alpha +1)}(x).
\end{align*}
\begin{itemize}
\item[3. ] Combine the results from Problem 1.2 to determine the three-term recurrence relationship and the  top $5 \times 5$ block of the Jacobi operator.

\end{itemize}
\rule{\textwidth}{1pt}
\subsection{Problem 4}
\begin{itemize}
\item[1. ] Represent the ordinary differential operator

\end{itemize}
\[
u'(x) - x u(x)  \qqfor x \geq 0
\]
as an operator on the coefficients of $u$ in a weighted Laguerre expansion:

\[
u(x) = \sum_{k=0}^\infty u_k \E^{-x/2} L_k(x) = \E^{-x/2} \bvect[L_0(x),L_1(x),\dots]\Vectt[u_0,u_1,\dots],
\]
where the range of the operator is specified in $\E^{-x/2} \bvect[L_0^{(1)}(x),L_1^{(1)}(x),\dots]$.

\begin{itemize}
\item[2. ] Show that the Laguerre polynomials are eigenfunctions of a Sturm\ensuremath{\endash}Liouville problem, that is, find  $\lambda_n^{(\alpha)}$ so that

\end{itemize}
\[
{\E^x \over x^\alpha} {\D \over \dx} \br[x^{\alpha+1} \E^{-x} {\D L_n^{(\alpha)} \over \dx}] = \lambda_n^{(\alpha)} L_n^{(\alpha)}(x)
\]
Re-express this as an ODE with polynomial coefficients.

\rule{\textwidth}{1pt}


\end{document}
